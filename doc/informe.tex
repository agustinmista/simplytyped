\documentclass[a4paper,10pt]{article}
\usepackage[utf8]{inputenc}
\usepackage{fancyhdr, float, graphicx, caption}
\usepackage{amsmath, amssymb}
\usepackage{bm}
\usepackage[margin=1in]{geometry}
\usepackage{multicol}
\usepackage{proof}
\usepackage{titlesec} 

\setlength{\inferLineSkip}{4pt}

\titleformat{\subsection}[runin]
  {\normalfont\large\bfseries}{\thesubsection}{1em}{}	
\titleformat{\subsubsection}[runin]
  {\normalfont\normalsize\bfseries}{\thesubsubsection}{1em}{}


\pagestyle{fancy}
\renewcommand{\figurename}{Figura}
\renewcommand\abstractname{\textit{Abstract}}

\fancyhf{}
\fancyhead[LE,RO]{\textit{Intérprete de Cálculo Lambda Simple Tipado}}
\fancyfoot[RE,CO]{\thepage}

%%%%%%%%%%%%%%%%%%%%%%%%%%%%%%%%%%%%%%%%%%%%%%%%%%%%%%%%%%%%

\title{
	%Logo UNR
	\begin{figure}[!h]
		\centering
		\includegraphics[scale=1]{unr.png}
		\label{}
	\end{figure}
	% Pie Logo
	\normalsize
		\textsc{Universidad Nacional de Rosario}\\	
		\textsc{Facultad de Ciencias Exactas, Ingeniería y Agrimensura}\\
		\textit{Licenciatura en Ciencias de la Computación}\\
		\textit{Análisis de Lenguajes de Programación}\\
	% Título
	\vspace{30pt}
	\hrule{}
	\vspace{15pt}
	\Large
		\textbf{Intérprete de Cálculo Lambda Simple Tipado}\\
	\vspace{15pt}
	\hrule{}
	\vspace{30pt}
	% Alumnos/docentes
	\begin{multicols}{2}
	\raggedright
		\large
			\textbf{Alumnos:}\\
		\normalsize
			CRESPO, Lisandro (C-6165/4) \\
			MISTA, Agustín (M-6105/1) \\
			$\;$ \\
			$\;$ \\
	\raggedleft
		\large
			\textbf{Docentes:}\\
		\normalsize
			JASKELIOFF, Mauro\\
			SIMICH, Eugenia\\
			MANZINO, Cecilia\\
			RABASEDAS, Juan Manuel\\
	\end{multicols}
}
%%%%%%%%%%%%%%%%%%%%%%%%%%%%%%%%%%%%%%%%%%%%%%%%%%%%%%%%%%%
\begin{document}
\date{14 de Octubre de 2015}
\maketitle

\pagebreak
%----------------------------------------------------------
\subsection*{Ejercicio 1} 
	\emph{Damos una derivación de tipo para el término \texttt{S} definido en Prelude.lam donde: 
		\linebreak 
		\\ 
		\indent $S = \lambda x\!:\!(B \rightarrow B \rightarrow B).\;\lambda y\!:\!(B\rightarrow B).\;\lambda z\!:\!B.\;\;(x\;z)\;(y\;z) \;:\; (B \rightarrow B \rightarrow B)\rightarrow(B \rightarrow B)\rightarrow B \rightarrow B$
	}
	\\
	\begin{align*}
		\infer[T_{ABS}]{
			\vdash \; \lambda x\!:\!(B \rightarrow B \rightarrow B).\;\lambda y\!:\!(B\rightarrow B).\;\lambda z\!:\!B.\;\;(x\;z)\;(y\;z) \;:\; (B \rightarrow B \rightarrow B)\rightarrow(B \rightarrow B)\rightarrow B \rightarrow B
		}{
			\infer[T_{ABS}]{
				x\!:\!(B \rightarrow B \rightarrow B) \; \vdash \; \lambda y\!:\!(B\rightarrow B).\;\lambda z\!:\!B.\;\;(x\;z)\;(y\;z) \;:\;(B \rightarrow B)\rightarrow B \rightarrow B
			}{
				\infer[T_{ABS}]{
					x\!:\!(B \rightarrow B \rightarrow B), \; y\!:\!(B\rightarrow B) \; \vdash \; \lambda z\!:\!B.\;\;(x\;z)\;(y\;z) \;:\; B\rightarrow B
				}{
					\infer[T_{ABS}]{
						x\!:\!(B \rightarrow B \rightarrow B), \; y\!:\!(B\rightarrow B),\;z\!:\!B \; \vdash \; (x\;z)\;(y\;z) \;:\; B
					}{
						\infer[T_{APP}]{
							\Gamma \; \vdash \; x\;z \;:\; B \rightarrow B
						}{
							\infer[T_{VAR}]{
								\Gamma \; \vdash \; x\!:\!(B \rightarrow B \rightarrow B)
							}{
								x\!:\!(B \rightarrow B \rightarrow B) \in \Gamma
							}
							\hspace{0.5cm}
							\infer[T_{VAR}]{
								\Gamma \; \vdash \; z\!:\!B
							}{
								x\!:\!B \in \Gamma
							}
						}
						\hspace{1cm}
						\infer[T_{APP}]{
							\Gamma \; \vdash \; y\;z \;:\; B
						}{
							\infer[T_{VAR}]{
								\Gamma \; \vdash \; y\!:\!(B \rightarrow B)
							}{
								y\!:\!(B \rightarrow B) \in \Gamma
							}\hspace{0.5cm}
							\infer[T_{VAR}]{
								\Gamma \; \vdash \; z\!:\!B
							}{
								z\!:\!B \in \Gamma
							}
						}
					}
				}
			}
		}
	\end{align*}
	\\
	\emph{Por comodidad, llamamos:\; $\Gamma = x\!:\!(B \rightarrow B \rightarrow B), \; y\!:\!(B\rightarrow B),\;z\!:\!B$}
	\\
%----------------------------------------------------------
\subsection*{Ejercicio 2} 
	\emph{Explicación y cositas}
	\\
%----------------------------------------------------------
\subsection*{Ejercicio 5} 
	\emph{Algo algo}
	\\
	\begin{align*}
		\infer[regla]{
			abajo
		}{
			arriba
		}
	\end{align*}
%----------------------------------------------------------
\subsection*{Ejercicio 7} 
	\emph{Algo algo}
	\\
	\begin{align*}
		\infer[regla1]{
			abajo
		}{
			arriba
		}
	\end{align*}
	\begin{align*}
		\infer[regla2]{
			abajo
		}{
			arriba
		}
	\end{align*}
%----------------------------------------------------------
\subsection*{Ejercicio 9} 
	\emph{Algo algo}
	\\
	\begin{align*}
		\infer[regla1]{
			abajo
		}{
			arriba
		}
	\end{align*}
%----------------------------------------------------------
\pagebreak  %Ver si es necesario
\\
\vspace{\fill}
\begin{multicols}{2}
	\hrule
	\vspace{5pt}
	CRESPO, Lisandro \\
	\linebreak
	\hrule
	\vspace{5pt}
	MISTA, Agustín \\
\end{multicols}

\end{document}
