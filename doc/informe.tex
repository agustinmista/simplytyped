\documentclass[a4paper,10pt]{article}
\usepackage[utf8]{inputenc}
\usepackage{fancyhdr, float, graphicx, caption}
\usepackage{amsmath, amssymb}
\usepackage{bm}
\usepackage[margin=1in]{geometry}
\usepackage{multicol}
\usepackage{proof}
\usepackage{titlesec}

\setlength{\inferLineSkip}{4pt}

\titleformat{\subsection}[runin]
  {\normalfont\large\bfseries}{\thesubsection}{1em}{}	
\titleformat{\subsubsection}[runin]
  {\normalfont\normalsize\bfseries}{\thesubsubsection}{1em}{}


\pagestyle{fancy}
\renewcommand{\figurename}{Figura}
\renewcommand\abstractname{\textit{Abstract}}

\fancyhf{}
\fancyhead[LE,RO]{\textit{Intérprete de Cálculo Lambda Simple Tipado}}
\fancyfoot[RE,CO]{\thepage}

%%%%%%%%%%%%%%%%%%%%%%%%%%%%%%%%%%%%%%%%%%%%%%%%%%%%%%%%%%%%

\title{
	%Logo UNR
	\begin{figure}[!h]
		\centering
		\includegraphics[scale=1]{unr.png}
		\label{}
	\end{figure}
	% Pie Logo
	\normalsize
		\textsc{Universidad Nacional de Rosario}\\	
		\textsc{Facultad de Ciencias Exactas, Ingeniería y Agrimensura}\\
		\textit{Licenciatura en Ciencias de la Computación}\\
		\textit{Análisis de Lenguajes de Programación}\\
	% Título
	\vspace{30pt}
	\hrule{}
	\vspace{15pt}
	\Large
		\textbf{Intérprete de Cálculo Lambda Simple Tipado}\\
	\vspace{15pt}
	\hrule{}
	\vspace{30pt}
	% Alumnos/docentes
	\begin{multicols}{2}
	\raggedright
		\large
			\textbf{Alumnos:}\\
		\normalsize
			CRESPO, Lisandro (C-6165/4) \\
			MISTA, Agustín (M-6105/1) \\
			$\;$ \\
			$\;$ \\
	\raggedleft
		\large
			\textbf{Docentes:}\\
		\normalsize
			JASKELIOFF, Mauro\\
			SIMICH, Eugenia\\
			MANZINO, Cecilia\\
			RABASEDAS, Juan Manuel\\
	\end{multicols}
}
%%%%%%%%%%%%%%%%%%%%%%%%%%%%%%%%%%%%%%%%%%%%%%%%%%%%%%%%%%%
\begin{document}
\date{14 de Octubre de 2015}
\maketitle

\pagebreak
%----------------------------------------------------------
\subsection*{Ejercicio 1} 
	\emph{Damos una derivación de tipo para el término \texttt{S} definido en Prelude.lam donde:
		\\
		\\ 
		\indent $S = \lambda x\!:\!B \rightarrow B \rightarrow B\; . \;\lambda y\!:\!B \rightarrow B\; . \;\lambda z\!:\!B\; . \;(x\;\;z)\;\;(y\;\;z)$
	}
	\\
	\\
	\begin{align*}
		\infer[T_{ABS}]{
			\vdash\lambda x\!:\!B \rightarrow B \rightarrow B\; . \;\lambda y\!:\!B \rightarrow B\; . \;\lambda z\!:\!B\; . \;(x\;\;z)\;\;(y\;\;z)\!:\!(B \rightarrow B \rightarrow B) \rightarrow (B \rightarrow B) \rightarrow B \rightarrow B
		}{
			\infer[T_{ABS}]{
				x\!:\!B \rightarrow B \rightarrow B \;\vdash\; \lambda y\!:\!B \rightarrow B\; . \;\lambda z\!:\!B\; . \;(x\;\;z)\;\;(y\;\;z)\!:\!(B \rightarrow B) \rightarrow (B \rightarrow B)
			}{
				\infer[T_{ABS}]{
					x\!:\!B \rightarrow B \rightarrow B, y\!:\!B \rightarrow B \;\vdash\;\lambda z\!:\!B\; . \;(x\;\;z)\;\;(y\;\;z)\!:\!B \rightarrow B
				}{
					\infer[T_{APP}]{
						x\!:\!B \rightarrow B \rightarrow B, y\!:\!B \rightarrow B, z\!:\!B \;\vdash\;(x\;\;z)\;\;(y\;\;z)\!:\!B
					}{
						\infer[T_{APP}]{
							\Gamma \;\vdash\;x\;\;z\!:\!B \rightarrow B
						}{
							\infer[T_{VAR}]{
								\Gamma \;\vdash\;x\!:\!B \rightarrow B \rightarrow B
							}{
								x\!:\!B \rightarrow B \rightarrow B \in \Gamma
							} & \qquad
							\infer[T_{VAR}]{
								\Gamma \;\vdash\;z\!:\!B
							}{
								z\!:\!B \in \Gamma
							}
						} & \qquad
						\infer[T_{APP}]{
							\Gamma \;\vdash\;y\;\;z\!:\!B
						}{
							\infer[T_{VAR}]{
								\Gamma \;\vdash\;y\!:\!B \rightarrow B
							}{
								y\!:\!B \rightarrow B \in \Gamma
							} & \qquad
							\infer[T_{VAR}]{
								\Gamma \;\vdash\;z\!:\!B
							}{
								z\!:\!B \in \Gamma
							}
						}
					}
				}
			}
		}
	\end{align*}
	\\
	\emph{Por comodidad, llamamos:\; $\Gamma = x\!:\!B \rightarrow B \rightarrow B, \; y\!:\!B\rightarrow B,\;z\!:\!B$}
	\\
	\\
	\\
%----------------------------------------------------------
\subsection*{Ejercicio 2} 
	\emph{}
	\\
	\\
	La función \texttt{infer} retorna un \texttt{Either String Type}, en lugar de un valor de tipo \texttt{Type}, ya que en caso de que la inferencia de tipo falle, retorna un \texttt{String} con detalles del error.
	\\
	El operador $>>=$ nos permite pasar valores no monádicos a funciones sin salir de una mónada. En nuestro caso, $>>=$, toma un \texttt{Either v} y una función \texttt{f} y retorna \texttt{Left v} si es una cadena y \texttt{f v} en otro caso. Esto es útil para la propagación de errores sin necesidad de hacer \emph{pattern matching} sobre los resultados previos.
	\\
	\\
	\\
%----------------------------------------------------------
\subsection*{Ejercicio 5} 
	\emph{Damos una derivación de tipo para el término:
		\\
		\\ 
		\indent $({\bf let}\;\;z=((\lambda x\!:\!B\;.\;x)\;\;{\bf as}\;\;B \rightarrow B)\;\;{\bf in}\;\;z)\;\;{\bf as}\;\;B \rightarrow B$
	}
	\\
	\\
	\begin{align*}
		\infer[T_{ASCRIBE}]{
			\vdash \; ({\bf let}\;\;z=((\lambda x\!:\!B\;.\;x)\;\;{\bf as}\;\;B \rightarrow B)\;\;{\bf in}\;\;z)\;\;{\bf as}\;\;B \rightarrow B\!:\!B \rightarrow B
		}{
			\infer[T_{LET}]
			{
				\vdash \; ({\bf let}\;\;z=((\lambda x\!:\!B\;.\;x)\;\;{\bf as}\;\;B \rightarrow B)\;\;{\bf in}\;\;z)\!:\!B \rightarrow B
			}{
				\infer[T_{ASCRIBE}]
				{
					\vdash \; (\lambda x\!:\!B\;.\;x)\;\;{\bf as}\;\;B \rightarrow B\!:\!B \rightarrow B
				}
				{
					\infer[T_{ABS}]
					{
						\vdash \; \lambda x\!:\!B\;.\;x\!:\!B \rightarrow B
					}
					{
						\infer[T_{VAR}]
						{
							x\!:\!B \; \vdash \; x\!:\!B
						}
						{
							x\!:\!B \; \in \; x\!:\!B
						}
					}
				} & \qquad
				\infer[T_{VAR}]
				{
					z\!:\!B \rightarrow B\; \vdash \;z\!:\!B \rightarrow B
				}
				{
					z\!:\!B \rightarrow B\; \in \;z\!:\!B \rightarrow B
				}
			}
		}
	\end{align*}
	\\
	\\
	\\
	
\pagebreak
%----------------------------------------------------------
\subsection*{Ejercicio 7} 
	\emph{Extendemos la relación de evaluación, agregando dos reglas nuevas:}
	\\
	\begin{align*}
		\infer[E_{TUP1}]{
			(t_{1},t_{2}) \rightarrow (t_{1}',t_{2})
		}{
			t_{1} \rightarrow t_{1}'
		} \hspace{3cm}
		\infer[E_{TUP2}]{
			(v,t_{2}) \rightarrow (v,t_{2}')
		}{
			t_{2} \rightarrow t_{2}'
		}
	\end{align*}
	\begin{align*}
		\infer[E_{FST1}]{
			{\bf fst}\;\;t \rightarrow {\bf fst}\;\;t'
		}{
			t \rightarrow t'
		} \hspace{3cm}
		\infer[E_{SND1}]{
			{\bf snd}\;\;t \rightarrow {\bf snd}\;\;t'
		}{
			t \rightarrow t'
		}
	\end{align*}
	\begin{align*}
		\infer[E_{FST2}]{
			{\bf fst}\;\;(t_{1},t_{2}) \rightarrow t_{1}
		}{
		} \hspace{3cm}
		\infer[E_{SND2}]{
			{\bf snd}\;\;(t_{1},t_{2}) \rightarrow t_{2}
		}{
		}
	\end{align*}
	\\
	\\
	\\	
%----------------------------------------------------------
\subsection*{Ejercicio 9} 
	\emph{Damos una derivación de tipo para el término:
		\\
		\\ 
		\indent $({\bf let}\;\;z=((\lambda x\!:\!B\;.\;x)\;\;{\bf as}\;\;B \rightarrow B)\;\;{\bf in}\;\;z)\;\;{\bf as}\;\;B \rightarrow B$
	}
	\\
	\\
	\begin{align*}
		\infer[T_{FST}]{
			\vdash \; {\bf fst}({\bf unit} \;{\bf as}\; {\bf Unit},\lambda x\!:\!(B,B)\;.\;{\bf snd}\;\;x)\!:\!{\bf Unit}
		}{
			\infer[T_{PAIR}]
			{
				\vdash \; ({\bf unit} \;{\bf as}\; {\bf Unit},\lambda x\!:\!(B,B)\;.\;{\bf snd}\;\;x)\!:\!({\bf Unit}, B)
			}
			{
				\infer[T_{ASCRIBE}]
				{
					\vdash \; {\bf unit}\;{\bf as}\;{\bf Unit}\!:\!{\bf Unit}
				}
				{
					\infer[T_{UNIT}]
					{
						\vdash \; {\bf unit}\!:\!{\bf Unit}
					}
					{}
				} & \qquad
				\infer[T_{ABS}]
				{
					\vdash \; \lambda x\!:\!(B,B) \;.\;{\bf snd}\;\;x\!:\!B
				}
				{
					\infer[T_{SND}]
					{
						x\!:\!(B,B) \; \vdash \; {\bf snd} \;\; x\!:\!B
					}
					{
						\infer[T_{VAR}]
						{
							x\!:\!(B,B) \; \vdash \; x\!:\!(B,B)
						}{
							x\!:\!(B,B) \in x\!:\!(B,B)
						}
					}
				}
			}
		}
	\end{align*}
%----------------------------------------------------------
\\
\vspace{\fill}
\begin{multicols}{2}
	\hrule
	\vspace{5pt}
	CRESPO, Lisandro \\
	\linebreak
	\hrule
	\vspace{5pt}
	MISTA, Agustín \\
\end{multicols}

\end{document}
